\documentclass[b]{subfiles}
\begin{document}
\chapter{Dilations}

\begin{parsec}[dils-intro]%
\begin{point}%
In this chapter we will study various dilation theorems.
The common theme is that a complicated map
    is actually the composition of a simpler map
    after a representation into a larger algebra.
We already saw a dilation theorem in disguise:
    using the Gel'fand--Naimark--Segal construction (see \sref{gns})
    one shows every state is a vector state on a larger algebra:
\end{point}
\begin{point}{Exercise (GNS')}%
    Show that
    for each pu-map~$\omega\colon \scrA \to \C$
        from a~C$^*$-algebra~$\scrA$
    there is a Hilbert space~$\scrH$,
    a miu-map~$\varrho\colon \scrA \to \scrB(\scrH)$
    and vector~$x \in \scrH$
    such that~$\omega = h \after \varrho$
    where~$h \colon \scrB(\scrH) \to \C$
    is given by~$h(T) = \left<Tx,x\right>$.
\end{point}

\begin{point}%
Probably the most famous dilation theorem is that
of Stinespring\cite[Thm.~1]{stinespring}.
\end{point}

\begin{point}{Theorem (Stinespring)}
    For every cp-map~$\varphi\colon \scrA \to \scrB (\scrH)$
    there is a Hilbert space~$\scrK$,
        miu-map~$\varrho\colon \scrA \to \scrB(\scrK)$
        and bounded operator~$V\colon \scrH \to \scrK$
        such that~$\varphi = \ad_V \after \varrho$,
        where~$\Define{\ad_V} \colon \scrB(\scrK) \to \scrB(\scrH)$
        is given by~$\ad_V(T):= V^*TV$.

Furthermore:
\begin{inparaenum}
\item
    if~$\varphi$ is normal, then~$\varrho$ is also normal \emph{and}
\item
    if~$\varphi$ is unital, then~$V$ is an isometry
        (or equivalently~$\ad_V$ is unital).
\end{inparaenum}
\end{point}

\begin{point}%
(We will see a detailed proof later in \TODO{}.)
Stinespring's theorem
is fundamental in the study
of quantum information and quantum computing:
it is used to prove entropy inequalities (e.g.~\cite{lindblad}),
bounds on optimal cloners (e.g.~\cite{werner}),
full completeness of quantum programming languages (e.g.~\cite{staton}),
security of quantum key distribution (e.g.~\cite{werner2}),
analyze quantum alternation (e.g.~\cite{prakash}),
to categorify quantum processes (e.g.~\cite{selinger}) \emph{and}
as an axiom to single out
quantum theory among information processing theories.\cite{chiribella}
A fair overview of all uses of Stinespring's theorem and its consequences
is a formidable task out of scope of this text.

Stinespring's theorem only applies
to maps of the form~$\scrA \to \scrB(\scrH)$
    and so do most of its usefull consequences.
One wonders:
    is there an extension of Stinespring's theorem
    to arbitrary np-maps~$\scrA \to \scrB$?
A different and less common question might be:
    is Stinespring's dilation categorical in some way.
That is: does it have a defining universal property?
Both questions turn out to be true:
Paschke's generalization of GNS for Hilbert C$^*$-modules\cite{paschke}
    turns out to have the same universal property
        as Stinespring's dilation and so it extends Stinespring
        to arbitrary np-maps.

First we will proof Stinespring's theorem.
Then we show it obeys a universal property.
Before we move on to Paschke's GNS
    we need to develop Paschke's theory of self-dual Hilbert C$^*$-modules.
Paschke's work builds on Sakai's characterization of von Neumann algebra's,
    which would take considerable effort to develop.
Thus to be as complete as possible
we avoid Sakai's characterization (in contrast to \cite{wwpaschke})
    and give new proofs
    of Paschke's results where required.
One major difference is that we will use
    a restricted completion (see \TODO{}) of a uniformity
    instead of considering the dual space of a Hilbert C$^*$-module.
\end{point}


\end{parsec}

\end{document}

% vim: ft=tex.latex
