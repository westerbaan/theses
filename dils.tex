\documentclass[b]{subfiles}
\begin{document}
\chapter{Dilations}

\begin{parsec}[dils-intro]%
\begin{point}%
In this chapter we will study various dilation theorems.
The common theme is that a complicated map
    is actually the composition of a simpler map
    after a representation into a larger algebra.
We already saw a dilation theorem in disguise:
    using the Gel'fand--Naimark--Segal construction (see \sref{gns})
    one shows every state is a vector state on a larger algebra:
\end{point}
\begin{point}{Exercise (GNS')}%
    Show that
    for each pu-map~$\omega\colon \scrA \to \C$
        from a~C$^*$-algebra~$\scrA$
    there is a Hilbert space~$\scrH$,
    a miu-map~$\varrho\colon \scrA \to \scrB(\scrH)$
    and vector~$x \in \scrH$
    such that~$\omega = h \after \varrho$
    where~$h \colon \scrB(\scrH) \to \C$
    is given by~$h(T) = \left<Tx,x\right>$.
\end{point}

\begin{point}%
Probably the most famous dilation theorem is that
of Stinespring\cite[Thm.~1]{stinespring}.
\end{point}

\begin{point}{Theorem (Stinespring)}
    For every cp-map~$\varphi\colon \scrA \to \scrB (\scrH)$
    there is a Hilbert space~$\scrK$,
        miu-map~$\varrho\colon \scrA \to \scrB(\scrK)$
        and bounded operator~$V\colon \scrH \to \scrK$
        such that~$\varphi = \ad_V \after \varrho$,
        where~$\Define{\ad_V} \colon \scrB(\scrK) \to \scrB(\scrH)$
        is given by~$\ad_V(T):= V^*TV$.

Furthermore:
\begin{inparaenum}
\item
    if~$\varphi$ is normal, then~$\varrho$ is also normal \emph{and}
\item
    if~$\varphi$ is unital, then~$V$ is an isometry
        (or equivalently~$\ad_V$ is unital).
\end{inparaenum}
\end{point}

\begin{point}%
(We will see a detailed proof later in \TODO{}.)
Stinespring's theorem
is fundamental in the study
of quantum information and quantum computing:
it is used to prove entropy inequalities (e.g.~\cite{lindblad}),
bounds on optimal cloners (e.g.~\cite{werner}),
full completeness of quantum programming languages (e.g.~\cite{staton}),
security of quantum key distribution (e.g.~\cite{werner2}),
analyze quantum alternation (e.g.~\cite{prakash}),
to categorify quantum processes (e.g.~\cite{selinger}) \emph{and}
as an axiom to single out
quantum theory among information processing theories.\cite{chiribella}
A fair overview of all uses of Stinespring's theorem and its consequences
is a formidable task out of scope of this text.

Stinespring's theorem only applies
to maps of the form~$\scrA \to \scrB(\scrH)$
    and so do most of its usefull consequences.
One wonders:
    is there an extension of Stinespring's theorem
    to arbitrary np-maps~$\scrA \to \scrB$?
A different and less common question might be:
    is Stinespring's dilation categorical in some way.
That is: does it have a defining universal property?
Both questions turn out to be true:
Paschke's generalization of GNS for Hilbert C$^*$-modules\cite{paschke}
    turns out to have the same universal property
        as Stinespring's dilation and so it extends Stinespring
        to arbitrary np-maps.

We start of this chapter with a detailed proof of Stinespring's theorem.
We continue to show it obeys a universal property.
Before we move on to Paschke's GNS
    we need to develop Paschke's theory of self-dual Hilbert C$^*$-modules.
Paschke's work builds on Sakai's characterization of von Neumann algebras,
    which would take considerable effort to develop in detail.
Thus to be as complete as possible
we avoid Sakai's characterization (in contrast to \cite{wwpaschke})
    and give new proofs
    of Paschke's results where required.
One major difference is that we will use
    a restricted completion (see \TODO{}) of a uniformity
    instead of considering the dual space of a Hilbert C$^*$-module.
We finish the first part of this chapter
    by constructing Paschke's dilation
    and establishing it indeed extends Stinespring's dilation.
In the second part of this chaper
    we prove several results about Paschke's dilations.
\TODO{more intro}
\end{point}
\end{parsec}

\begin{parsec}[dils-stinespring]%
\begin{point}%
For the GNS-construction
it was necessary to ``complete''\footnote{Note that
        we do not require an inner product to be definite.
    The inner product on a Hilbert space \emph{is} definite.
    Thus the completion will quotient out those vectors with
        zero norm with respect to the inner product.}
    a complex vector space with inner product into a Hilbert space.
This completion was only sketched in \sref{completion-inner-product-space}.
Here we will work through the details
    as the corresponding completion required
    in Paschke's dilation
    is more complex and its exposition will benefit
    from this familiar analogon.
\end{point}

\begin{point}{Proposition}%
    Let~$V$ be a complex vector space with inner
        product~$[\,\cdot\,,\,\cdot\,]$.
    There is a Hilbert space~$\scrH$
        together with bounded linear map~$\eta\colon V \to \scrH$
            such that
        \begin{inparaenum}
        \item
        $[v,w] = \left<\eta(v), \eta(w)\right>$
            for all~$v,w \in V$ and
        \item
        the image of~$\eta$ is dense in~$\scrH$.
        \end{inparaenum}
\begin{point}{Proof}%
We will form~$\scrH$ from the set of Cauchy sequences in~$V$
    with a little twist.
Recall two Cauchy
    sequences~$(v_n)_n$ and~$(w_n)_n$ in~$V$
    are said to be equivalent
    if for every~$\varepsilon > 0$
    there is a~$n_0$
    such that~$\| v_n - w_n \| \leq \varepsilon$
    for all~$n \geq n_0$,
    where $\|v\| \equiv \sqrt{[v,v]}$.
Call a Cauchy sequence~$(v_n)_n$ \Define{fast}
    if for each~$n_0$
    we have~$\| v_n - v_m\| \leq \frac{1}{2^{n_0}}$
    for all~$n,m \geq n_0$.
Clearly every Cauchy sequence has a fast subsequence
    which is (as are all subsequences) equivalent with it.
\begin{point}%
    Define~$\scrH$ to be the set of fast Cauchy sequences modulo
        equivalence.
To define the inner product on~$\scrH$,
first note that for Cauchy sequences~$(v_n)_n$ and~$(w_n)_n$ in~$V$
the sequence~$([v_n,w_n])_n$ is Cauchy in~$\C$.
Indeed as with Cauchy--Schwarz (\sref{chilb-cs}) we have
\begin{align*}
    \bigl|[v_n,w_n] - [v_m,w_m]\bigr|
    & \ =\  \bigl|[v_n,w_n-w_m] + [v_n - v_m,w_m]\bigr| \\
    & \ \leq\  \|v_n\| \|w_n - w_m\| + \|v_n-v_m\|\|w_m\|,
\end{align*}
and Cauchy sequences are bounded
\end{point}
\end{point}
\end{point}
\end{parsec}

\end{document}

% vim: ft=tex.latex
