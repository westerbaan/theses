\documentclass[b]{subfiles}
\begin{document}
\chapter{Dilations}

\begin{parsec}[dils-intro]%
\begin{point}%
In this chapter we will study various dilation theorems.
The common theme is that a complicated map
    is actually the composition of a simpler map
    after a representation into a larger algebra.
We already saw a dilation theorem in disguise:
    using the Gel'fand--Naimark--Segal construction (see \sref{gns})
    one shows every state is a vector state on a larger algebra:
\end{point}
\begin{point}{Exercise (GNS')}%
    Show that
    for each pu-map~$\omega\colon \scrA \to \C$
        from a~C$^*$-algebra~$\scrA$
    there is a Hilbert space~$\scrH$,
    a miu-map~$\varrho\colon \scrA \to \scrB(\scrH)$
    and vector~$x \in \scrH$
    such that~$\omega = h \after \varrho$
    where~$h \colon \scrB(\scrH) \to \C$
    is given by~$h(T) = \left<Tx,x\right>$.
\end{point}

\begin{point}%
Probably the most famous dilation theorem is that
of Stinespring\cite[Thm.~1]{stinespring}.
\end{point}

\begin{point}{Theorem (Stinespring)}
    For every cp-map~$\varphi\colon \scrA \to \scrB (\scrH)$
    there is a Hilbert space~$\scrK$,
        miu-map~$\varrho\colon \scrA \to \scrB(\scrK)$
        and bounded operator~$V\colon \scrH \to \scrK$
        such that~$\varphi = \ad_V \after \varrho$,
        where~$\Define{\ad_V} \colon \scrB(\scrK) \to \scrB(\scrH)$
        is given by~$\ad_V(T):= V^*TV$.

Furthermore:
\begin{inparaenum}
\item
    if~$\varphi$ is normal, then~$\varrho$ is also normal \emph{and}
\item
    if~$\varphi$ is unital, then~$V$ is an isometry
        (or equivalently~$\ad_V$ is unital).
\end{inparaenum}
\end{point}


\end{parsec}

\end{document}

% vim: ft=tex.latex
