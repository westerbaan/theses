\documentclass[main]{subfiles}
\begin{document}
\begin{parsec}%
\begin{point}{Definition}%
A \define{$C^*$-algebra}
is a complex vector space~$\scrA$
endowed with
\begin{enumerate}
\item
a binary operation,
called \define{multiplication}
(and denoted as such),
which is associative, and linear in both coordinates;
\item
an element~$1$, called \define{unit},
such that $1\cdot a = a = a\cdot 1$
for all~$a\in \scrA$;
\item
a unary operation $(\,\cdot\,)^*$,
called \define{involution},
such that $(a^*)^*=a$,
$(ab)^*=b^*a^*$,
$(\lambda a)^* = \bar\lambda a^*$,
and $(a+b)^* = a^*+b^*$
for all~$a,b\in\scrA$ and~$\lambda\in \C$;
\item
a complete \define{norm} $\|\,\cdot\,\|$
such that
$\|ab\|\leq\|a\|\|b\|$
for all~$a,b\in\scrA$,
and 
\begin{equation*}
\label{eq:Cstar-identity}
\|a^*a\|\ =\ \|a\|^2
\end{equation*}
holds, which is called the \define{$C^*$-identity}.
\end{enumerate}
The $C^*$-algebra $\scrA$ is called \define{commutative}
if $ab=ba$ for all~$a,b\in\scrA$.
\end{point}
\end{parsec}
\begin{parsec}%
\begin{point}{Examples}%
The set~$\C$ of \define{complex numbers}
forms a commutative  $C^*$-algebra
in which addition, (scalar) multiplication, and~$1$
have their usual meaning.
Involution is given by conjugation ($z^*=\bar{z}$),
and norm by modulus ($\|z\|=|z|$).
\begin{point}%
Let~$X$ be a compact Hausdorff space.
The set $\define{C(X)}$ of \define{continuous functions}
from~$\scrA$ to~$\C$
forms a commutative $C^*$-algebra
when addition, (scalar) multiplication, involution and~$1$ are
interpretted coordinatewise \grayed{(e.g.~$(f+g)(x)=f(x)+g(x)$)},
and the norm is taken to be 
$\|f\|=\sup_{x\in X} |f(x)|$
(the \define{sup-norm}).
\begin{point}%
The set~$\scrB(\scrH)$ of bounded operators
on a Hilbert space~$\scrH$ forms a $C^*$-algebra.
\end{point}
\end{point}
\end{point}
\end{parsec}
\begin{parsec}%
\begin{point}{Definition}
An element $a$ of a $C^*$-algebra $\scrA$ is called
\begin{enumerate}
\item \define{self-adjoint} if $a^* =a$;
\item \define{positive}
if $a\equiv b^*b$ for some $b\in \scrA$;
\item a \define{projection} if $a=a^*a$;
\item \define{central} if $ab=ba$ for all~$b\in\scrA$.
\end{enumerate}
\end{point}
\end{parsec}
%
% morphisms between C^*-algebras
%
\begin{parsec}
\begin{point}{Definition}
A linear map $f\colon \scrA \to \scrB$
between $C^*$-algebras
is called
\begin{enumerate}
\item
\define{\textbf{m}ultiplicative}
if $f(ab)=f(a)f(b)$ for all $a,b\in\scrA$;
\item
\define{\textbf{i}nvolutive}
if $f(a^*)=f(a)^*$ for all~$a\in\scrA$;
\item
\define{\textbf{p}ositive}
if $f(a)$ is positive
for every positive $a\in\scrA$, and
\item
\define{\textbf{u}nital}
if $f(1)=1$.
\end{enumerate}
\end{point}
\begin{point}%
We use the bold letters as abbreviations,
so for instance,
$f$ is \define{pu} if it is positive and unital,
and a \define{miu-map}
is a multiplicative, involutive, unital linear map between $C^*$-algebras,
(which is usually called a \define{unital $*$-homomorphism}.)
\end{point}
\end{parsec}

\begin{parsec}[gelfand]%
\begin{point}{Gelfand's Representation Theorem}%
Let~$\scrA$ be a commutative $C^*$-algebra.
\begin{enumerate}
\item
The \define{spectrum} of~$\scrA$, \define{$\spec(\scrA)$},
the set 
of miu-maps $\scrA\to \C$
endowed with the topology of pointwise convergence,
is a compact Hausdorff space.

\item
The \define{Gelfand transform},
the map $\gamma\colon \scrA\to C(\spec(\scrA))$
given by $\gamma(a)(\varphi)=\varphi(a)$,
is an isomorphism.
\end{enumerate}
\end{point}
\end{parsec}

\begin{parsec}[cstar-square]%
\begin{point}{Corollary}%
For each positive element~$a$ of a $C^*$-algebra~$\scrA$,
there is a unique positive element~\define{$\sqrt{a}$} of~$\scrA$
with $(\sqrt{a})^2 = a$.

\begin{point}[cstar-square-commutes]%
Moreover,
if $b\in \scrA$ commutes with~$a$,
then~$b$ commutes with~$\sqrt{a}$.
\end{point}
\end{point}
\end{parsec}
%
%
%
\begin{parsec}[ad-monotone]%
\begin{point}{Exercise}%
Prove that $b\leq c\implies a^*ba \leq a^*ca$
for $a,b,c$ from a $C^*$-algebra.
\end{point}
\end{parsec}
%
% Suprema in a von Neumann algebra
%
\begin{parsec}[vna-suprema]%
\begin{point}{Lemma}%
Let~$D$ be a bounded directed set of self-adjoint elements
of a von Neumann algebra~$\scrA$
(so $D\subseteq \sa{\scrA}$).
Then~$D$ has a supremum, $\bigvee D$
(in $\sa{\scrA}$.)
\begin{point}[vna-supremum-commutes]%
Moreover, 
if $a\in\scrA$ commutes with all~$d\in D$,
then $a$ commutes with~$\bigvee D$.
\end{point}
\end{point}
\begin{point}{Proof}%
\TODO{}
\end{point}
\end{parsec}
%
%
%
\begin{parsec}[ad-normal]%
\begin{point}{Proposition}%
Let~$a$ be an element of a von Neumann algebra~$\scrA$.
Then~$\bigvee_{d\in D} a^*\,d\,a = a^*\,(\bigvee D)\, a$
for every bounded directed subset~$D$ of self-adjoint
elements of~$\scrA$.
\end{point}
\begin{point}{Proof}%
If~$a$ is invertible,
then the (by~\sref{ad-monotone}) order preserving map $b\mapsto a^*ba$
has an order preserving inverse (namely $b\mapsto (a^{-1})^* b a^{-1}$),
and therefor preserves all suprema.
\begin{point}%
The general case (in which~$a$ need not be invertible)
requires more effort.


\end{point}
\end{point}
\end{parsec}



\begin{parsec}[ad-contraposed]%
\begin{point}{Lemma}%
Let~$a$ be an element of a $C^*$-algebra~$\scrA$
with $\|a\|\leq 1$,
and let~$p$ and~$q$ be projections on~$\scrA$.
Then 
$a^* p a \leq q^\perp$
iff $paq=0$
iff  $aqa^*\leq p^\perp$.
\end{point}
\begin{point}{Proof}%
Suppose that~$a^*pa\leq q^\perp$.
Then we have $q a^*pa q \leq qq^\perp q = 0$
(see \sref{ad-monotone})
and so $paq=0$,
because $\|paq\|^2=\|(paq)^*paq\|=0$
by the $C^*$-identity.
Applying $(\,\cdot\,)^*$,
we get $qa^*p=0$, and so both $qa^* = qa^*p^\perp$
and $aq = p^\perp aq$, giving
us $aqa^* = p^\perp a q a^* p^\perp 
\leq p^\perp$,
where we used that $aqa^*\leq aa^*\leq \|aa^*\|=\|a\|^2\leq 1$.
By a similar reasoning,
we get $aqa^*\leq p^\perp \implies paq=0\implies a^*pa\leq q^\perp$.
\end{point}
\begin{point}[projection-above-effect]{Corollary}%
Let~$a$ be an effect of a $C^*$-algebra~$\scrA$,
and let~$p$ be a projection from~$\scrA$.
Then $a\leq p$ iff $ap=p$ iff $pa=p$ iff $ap^\perp=0$ iff $p^\perp a=0$.
\end{point}
\end{parsec}

%
% Projections
%
\begin{parsec}%
\begin{point}{Proposition}%
Above every effect~$b$ of a von Neumann algebra~$\scrA$,
there is a smallest projection, \define{$\ceil{b}$},
called the \define{ceiling} of~$b$,
 given by $\ceil{b}=\bigvee_{n=0}^\infty b^{\nicefrac{1}{2^n}}$.
\begin{point}[vna-ceil-commutes]%
Moreover, if $a\in \scrA$ commutes with $b$,
then~$a$ commutes with~$\ceil{b}$.
\end{point}
\end{point}
\begin{point}{Proof}
Let~$p$ denote the supremum of~$0\leq b\leq b^{\nicefrac{1}{2}}\leq
b^{\nicefrac{1}{4}}\leq\dotsb\leq 1$.

\begin{point}[vna-ceil-point-1]%
To begin,
note that if~$a\in \scrA$
commutes with~$b$,
then~$a$ commutes with~$p$.
Indeed, for such~$a$ we have~$a\sqrt{b}=\sqrt{b}a$
by~\sref{cstar-square-commutes},
and so $a b^{\nicefrac{1}{2^n}} = b^{\nicefrac{1}{2^n}} a$
for each~$n$
by induction.
Thus~$ap=pa$ by~\sref{vna-supremum-commutes}.
\end{point}
\begin{point}%
To see that~$p$ is a projection,
it suffices 
(by~\sref{gelfand})
to show that~$p^3=p$
since~$p$ is positive.
Note that, because~$b^{\nicefrac{1}{2}}=\sqrt{b}$
commutes with~$b$,
we see that~$\sqrt{b}$ commutes 
with~$p$ (by~\sref{vna-ceil-point-1}.)
Thus $bp = \sqrt{b}\,p\,\sqrt{b}
=\bigvee_{n=0}^\infty \sqrt{b} \,b^{\nicefrac{1}{2^n}}\,\sqrt{b} = p$
by~\sref{ad-normal}.
Similarly, for each~$n$,
we get $b^{\nicefrac{1}{2^n}} p = p = p b^{\nicefrac{1}{2^n}}$,
thus $ p\, b^{\nicefrac{1}{2^n}}\,p=p$,
and so $p^3 = \bigvee_n p\,b^{\nicefrac{1}{2^n}} \,p = p$.
Thus~$p$ is a projection.
\end{point}
\begin{point}%
It remains to be shown that~$p$ is the \emph{least} projection
above~$b$.
Let~$q$ be a projection in~$\scrA$ with $b\leq q$;
we must show that~$q\leq p$.
Since~$q$ commutes with~$b$ (by~\sref{projection-above-effect}),
we have $b^{\nicefrac{1}{2}}\leq q$ 
(by Gelfand's Representation Theorem, \sref{gelfand}.)
Since by induction $b^{\nicefrac{1}{2^n}}\leq q$ for all~$n$,
we get $p\leq q$ by definition of~$p$.
\end{point}
\end{point}
\end{parsec}



\end{document}
